\documentclass{beamer}
\usepackage{etex}

\usepackage[utf8]{inputenc}
\usepackage[frenchb]{babel}
\usepackage{verbatim}
\usepackage{graphicx}
\usepackage{color}
\usepackage{hyperref}
\usepackage{verbatim}
\usepackage{url}
\usepackage{moreverb}
\usepackage{fancyvrb}
\usepackage{minted}
\usepackage{natbib}
\usepackage{eulervm}
\usepackage{auto-pst-pdf}
\usepackage{pst-plot}
\usepackage{amssymb}% http://ctan.org/pkg/amssymb
\usepackage{pifont}% http://ctan.org/pkg/pifont
\usepackage{tikz}
\newcommand{\cmark}{\ding{51}}%
\newcommand{\xmark}{\ding{55}}%

\usetikzlibrary{shapes,arrows}
\tikzset{
  every overlay node/.style={
    draw=white,fill=white,anchor=north west,
  },
  every overlay node border/.style={
    draw=black,fill=white,anchor=north west,rounded corners,
  },
}
% Usage:
% \tikzoverlay at (-1cm,-5cm) {content};
% or
% \tikzoverlay[text width=5cm] at (-1cm,-5cm) {content};
\def\tikzoverlay{%
   \tikz[baseline,overlay]\node[every overlay node]
}%
\def\tikzoverlayborder{%
   \tikz[baseline,overlay]\node[every overlay node border]
}%

\newrgbcolor{mygreen}{.00 .5 .00}
\newcommand{\X}[1]{\textcolor{blue}{#1}}
\newcommand{\y}[1]{\textcolor{red}{#1}}
\newcommand{\model}[1]{\textcolor{mygreen}{#1}}
\newcommand{\loss}[1]{\textcolor{lightblue}{#1}}

\hypersetup{colorlinks=true, linkcolor=black, urlcolor=blue}
\usetheme{boxes}
\beamertemplatenavigationsymbolsempty
\setbeamertemplate{sections/subsections in toc}[circle]
\setbeamertemplate{footline}[frame number]
\setbeamertemplate{itemize items}[circle]
\setbeamertemplate{itemize subitem}[square]

\title{{\bf Tree models with Scikit-Learn}\\
Great learners with little assumptions}
\author{Gilles Louppe (\href{https://twitter.com/glouppe}{@glouppe}, CERN)}
\date{April 3, 2015}

\newcommand{\todo}[1]{\textcolor{red}{[TODO] #1}}

\DeclareMathOperator*{\argmax}{arg\,max}

\begin{document}

\begin{frame}
\titlepage
\end{frame}


% Outline =====================================================================

\begin{frame}
  \frametitle{Outline}
  %\tableofcontents
  \setbeamertemplate{enumerate items}[circle]
  \begin{enumerate}
  \item Motivation

  \vspace{0.5cm}

  \item Growing decision trees

  \vspace{0.5cm}

  \item Random forests

    \vspace{0.5cm}

  \item Boosting

  \vspace{0.5cm}

  \item Reading tree leaves

  \vspace{0.5cm}

  \item Summary
  \end{enumerate}
\end{frame}


% Motivation ==================================================================

\section{Motivation}

\begin{frame}
  \frametitle{Motivation}
\end{frame}

% use cases
% running example


% Growing decision trees ======================================================

\section{Growing decision trees}

\begin{frame}
  \frametitle{Growing decision trees}
\end{frame}

% supervised learning framework (using scikit-learn notations)
% performance evaluation
% divide and conquer
% framework impurity / simple model at leaves => mention density estimation


% Forests and boosting ========================================================

\section{Random Forests}

\begin{frame}
  \frametitle{Random Forests}
\end{frame}

% Bias-variance decomposition
% Random forests


% Forests and boosting ========================================================

\section{Boosting}

\begin{frame}
  \frametitle{Boosting}
\end{frame}

% Framework (loss function, regularization, etc)


% Reading tree leaves =========================================================

\section{Reading tree leaves}

\begin{frame}
  \frametitle{Reading tree leaves}
\end{frame}

% Visualizing trees
% Variable importances
% Partial dependence plots
% Outlier detection
% Embedding (bag of leaves)


% Summary =================================================================

% strong points and weaknesses
% going further


\section{Summary}

\begin{frame}
  \frametitle{Summary}
\end{frame}

\begin{frame}
\begin{center}
{\Huge  Questions?}
\end{center}
\end{frame}

\end{document}
